%%%%%%%%%%%%%%%%%%%%%%%%%%%%%%%%%%%%%%%
% Deedy - One Page Two Column Resume
% LaTeX Template
% Version 1.1 (30/4/2014)
%
% Original author:
% Debarghya Das (http://debarghyadas.com)
%
% Original repository:
% https://github.com/deedydas/Deedy-Resume
%
% IMPORTANT: THIS TEMPLATE NEEDS TO BE COMPILED WITH XeLaTeX
%
% This template uses several fonts not included with Windows/Linux by
% default. If you get compilation errors saying a font is missing, find the line
% on which the font is used and either change it to a font included with your
% operating system or comment the line out to use the default font.
% 
%%%%%%%%%%%%%%%%%%%%%%%%%%%%%%%%%%%%%%
% 
% TODO:
% 1. Integrate biber/bibtex for article citation under publications.
% 2. Figure out a smoother way for the document to flow onto the next page.
% 3. Add styling information for a "Projects/Hacks" section.
% 4. Add location/address information
% 5. Merge OpenFont and MacFonts as a single sty with options.
% 
%%%%%%%%%%%%%%%%%%%%%%%%%%%%%%%%%%%%%%
%
% CHANGELOG:
% v1.1:
% 1. Fixed several compilation bugs with \renewcommand
% 2. Got Open-source fonts (Windows/Linux support)
% 3. Added Last Updated
% 4. Move Title styling into .sty
% 5. Commented .sty file.
%
%%%%%%%%%%%%%%%%%%%%%%%%%%%%%%%%%%%%%%%
%
% Known Issues:
% 1. Overflows onto second page if any column's contents are more than the
% vertical limit
% 2. Hacky space on the first bullet point on the second column.
%
%%%%%%%%%%%%%%%%%%%%%%%%%%%%%%%%%%%%%%

\documentclass[]{deedy-resume-openfont} 
\begin{document}

%%%%%%%%%%%%%%%%%%%%%%%%%%%%%%%%%%%%%%
%
%     LAST UPDATED DATE
%
%%%%%%%%%%%%%%%%%%%%%%%%%%%%%%%%%%%%%%
% \lastupdated

%%%%%%%%%%%%%%%%%%%%%%%%%%%%%%%%%%%%%%
%
%     TITLE NAME
%
%%%%%%%%%%%%%%%%%%%%%%%%%%%%%%%%%%%%%%


\namesection{}{Tianyu Cheng}{ \urlstyle{same}\url{http://tycheng.github.io} \\
\href{mailto:tianyu.cheng@utexas.edu}{tianyu.cheng@utexas.edu} | Phone (512).517.1107
}

%%%%%%%%%%%%%%%%%%%%%%%%%%%%%%%%%%%%%%
%
%     COLUMN ONE
%
%%%%%%%%%%%%%%%%%%%%%%%%%%%%%%%%%%%%%%

\begin{minipage}[t]{0.33\textwidth} 

%%%%%%%%%%%%%%%%%%%%%%%%%%%%%%%%%%%%%%
%     EDUCATION
%%%%%%%%%%%%%%%%%%%%%%%%%%%%%%%%%%%%%%

\section{Education} 

\subsection{University of Texas}

\vspace{\topsep} % Hacky fix for awkward extra vertical space

\descript{M.S. in Computer Science}
\location{May 2017 | Austin, TX}
College of Natural Science \\
Five Years BS/MS Integrated Program \\
\location{Major GPA: 3.9 / 4.0}
\sectionsep

\descript{B.S. in Computer Science}
\location{May 2016 | Austin, TX}
College of Natural Science \\
Turing Scholars Program \\
\location{Major GPA: 3.9 / 4.0}
\sectionsep


%%%%%%%%%%%%%%%%%%%%%%%%%%%%%%%%%%%%%%
%     COURSEWORK
%%%%%%%%%%%%%%%%%%%%%%%%%%%%%%%%%%%%%%

\section{Courses}

\subsection{Undergraduate}
Algorithm \& Complexity \\
Computer Vision \\
Artificial Intelligence \\
Programming Languages \\
Operating Systems \\
Data Management \\
Data Structure \\

\sectionsep

\subsection{Graduate}
Compilers \\
Computer Graphics \\
Autonomous Robots \\
Software Design \\

\sectionsep

%%%%%%%%%%%%%%%%%%%%%%%%%%%%%%%%%%%%%%
%     SKILLS
%%%%%%%%%%%%%%%%%%%%%%%%%%%%%%%%%%%%%%

\section{Skills}
\subsection{Programming}
\begin{tabular}{ll}
\skillbar{C/C++}{0.7}
\skillbar{Java}{0.8}
\skillbar{C\#}{0.5}
\skillbar{Python}{0.6}
\end{tabular}

\sectionsep

\subsection{Web Development}
\begin{tabular}{ll}
\skillbar{HTML/CSS}{0.7}
\skillbar{JavaScript}{0.6}
\skillbar{Node.js}{0.5}
\skillbar{Django}{0.5}
\end{tabular}

\sectionsep

\subsection{Computer Graphics}
\begin{tabular}{ll}
\skillbar{OpenGL}{0.7}
\skillbar{WebGL}{0.6}
\skillbar{GLSL}{0.6}
\end{tabular}

\sectionsep

%%%%%%%%%%%%%%%%%%%%%%%%%%%%%%%%%%%%%%
%     LINKS
%%%%%%%%%%%%%%%%%%%%%%%%%%%%%%%%%%%%%%

\section{Links} 
Github: \href{https://github.com/tycheng}{\custombold{tycheng}} \\
% LinkedIn: \href{www.linkedin.com/pub/tianyu-cheng/66/335/834/}{\custombold{tianyu-cheng}} \\
Homepage: \href{http://tycheng.github.io}{\custombold{tycheng.github.io}} \\

\sectionsep

%%%%%%%%%%%%%%%%%%%%%%%%%%%%%%%%%%%%%%
%
%     COLUMN TWO
%
%%%%%%%%%%%%%%%%%%%%%%%%%%%%%%%%%%%%%%

\end{minipage} 
\hfill
\begin{minipage}[t]{0.66\textwidth} 

%%%%%%%%%%%%%%%%%%%%%%%%%%%%%%%%%%%%%%
%     EXPERIENCE
%%%%%%%%%%%%%%%%%%%%%%%%%%%%%%%%%%%%%%

\section{Experience}

\runsubsection{Apple}
\descript{| GPU Validation Team }
\location{June 2016 – August 2016 | Austin, TX}
\vspace{\topsep} % Hacky fix for awkward extra vertical space
\begin{tightemize} 
\item developed an internal web front-end tool for performance visualization
\item implemented and validated counters in performance model
\item worked on numerics validation
\end{tightemize}
\sectionsep

\runsubsection{Apple}
\descript{| GPU Validation Team }
\location{June 2015 – August 2015 | Austin, TX}
% \vspace{\topsep} % Hacky fix for awkward extra vertical space
\begin{tightemize} 
\item developed an internal server-side tool with Ruby on Rails for test automation
\item developed a web front-end data analysis tool for data visualization
\item worked on numerics validation
\end{tightemize}
\sectionsep

\runsubsection{Digital Media Institute}
\descript{| Student Technician }
\location{June 2014 – December 2014 | Austin, TX}
% \vspace{\topsep} % Hacky fix for awkward extra vertical space
\begin{tightemize} 
\item worked on the back-end OOP design and implementation of an educational game with Unity and C\#
\item developed several third-party tools to facilitate game data management in Python, and provides a sanity check of the validity of the data
\item refactored back-end codes to comply with MVC pattern
\end{tightemize}
\sectionsep

%%%%%%%%%%%%%%%%%%%%%%%%%%%%%%%%%%%%%%
%     Projects
%%%%%%%%%%%%%%%%%%%%%%%%%%%%%%%%%%%%%%

\section{Projects}

\runsubsection{Ray Tracer}
\descript{| Computer Graphics }
\begin{tightemize} 
\item a multithreaded ray tracer based on Whitted model
\item used KD-tree and SAH for ray-object intersection optimization
\item supports glossiness and depth of field using distribution ray tracing
\end{tightemize}
\sectionsep

\runsubsection{GameL}
\descript{| Scala DSL }
\begin{tightemize} 
\item a game scripting DSL(domain-specific language) using Scala and Swing
\item designed and implemented a set of syntax for basic game object manipulation
\item attaches a demo of the classical game Snake using GameL
\end{tightemize}
\sectionsep

\runsubsection{3D Model Viewer}
\descript{| OpenGL GLSL}
\begin{tightemize} 
\item a shader-based OpenGL program that renders 3D models of format PMD/PMX(Polygon Model Data/eXtend) with simple animation
\item currently being ported to web platform using WebGL and CoffeeScript
\end{tightemize}
\sectionsep

\runsubsection{Online Linear Algebra Solver}
\descript{| Python Django}
\begin{tightemize} 
\item a web project aiming at teaching students linear algebra by example
\item solve linear algebra problems and show the individual steps,
    including row reduction, matrix multiplication, inverse of matrices, etc
\end{tightemize}
\sectionsep

\runsubsection{Online Website Designer}
\descript{| UI \& UX Design }
\begin{tightemize} 
\item a Java/Struts web project for UI/UX design
\item provides a user-friendly interface to customize websites by drag\&drop
\item CREDIT: This project owes the inspiration to online website editing tools, e.g. Weebly and Yola. 
\end{tightemize}
\sectionsep

\end{minipage} 
\end{document}  \documentclass[]{article}
