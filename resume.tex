%%%%%%%%%%%%%%%%%%%%%%%%%%%%%%%%%%%%%%%
% Deedy - One Page Two Column Resume
% LaTeX Template
% Version 1.1 (30/4/2014)
%
% Original author:
% Debarghya Das (http://debarghyadas.com)
%
% Original repository:
% https://github.com/deedydas/Deedy-Resume
%
% IMPORTANT: THIS TEMPLATE NEEDS TO BE COMPILED WITH XeLaTeX
%
% This template uses several fonts not included with Windows/Linux by
% default. If you get compilation errors saying a font is missing, find the line
% on which the font is used and either change it to a font included with your
% operating system or comment the line out to use the default font.
%
%%%%%%%%%%%%%%%%%%%%%%%%%%%%%%%%%%%%%%
%
% TODO:
% 1. Integrate biber/bibtex for article citation under publications.
% 2. Figure out a smoother way for the document to flow onto the next page.
% 3. Add styling information for a "Projects/Hacks" section.
% 4. Add location/address information
% 5. Merge OpenFont and MacFonts as a single sty with options.
%
%%%%%%%%%%%%%%%%%%%%%%%%%%%%%%%%%%%%%%
%
% CHANGELOG:
% v1.1:
% 1. Fixed several compilation bugs with \renewcommand
% 2. Got Open-source fonts (Windows/Linux support)
% 3. Added Last Updated
% 4. Move Title styling into .sty
% 5. Commented .sty file.
%
%%%%%%%%%%%%%%%%%%%%%%%%%%%%%%%%%%%%%%%
%
% Known Issues:
% 1. Overflows onto second page if any column's contents are more than the
% vertical limit
% 2. Hacky space on the first bullet point on the second column.
%
%%%%%%%%%%%%%%%%%%%%%%%%%%%%%%%%%%%%%%

\documentclass[]{deedy-resume-openfont}
\begin{document}

%%%%%%%%%%%%%%%%%%%%%%%%%%%%%%%%%%%%%%
%
%     LAST UPDATED DATE
%
%%%%%%%%%%%%%%%%%%%%%%%%%%%%%%%%%%%%%%

%%%%%%%%%%%%%%%%%%%%%%%%%%%%%%%%%%%%%%
%
%     TITLE NAME
%
%%%%%%%%%%%%%%%%%%%%%%%%%%%%%%%%%%%%%%

\newpage
\vspace{-10pt}
\centering{
    \fontspec[Path = fonts/lato/]{Lato-Hai}\fontsize{24pt}{0cm}\selectfont Tianyu
    \fontspec[Path = fonts/lato/]{Lato-Lig}\selectfont Cheng
}
\vspace{5pt} \\
\centering{ \color{headings}\fontspec[Path = fonts/raleway/]{Raleway-Medium}\fontsize{11pt}{14pt}\selectfont
    \href{mailto:tianyu_cheng@apple.com}{tianyu\_cheng@apple.com}
    (512).517.1107
}
\centering{
    \noindent\makebox[\linewidth]{\rule{\paperwidth}{0.4pt}}
}
\vspace{-20pt}


%%%%%%%%%%%%%%%%%%%%%%%%%%%%%%%%%%%%%%
%
%     COLUMN ONE
%
%%%%%%%%%%%%%%%%%%%%%%%%%%%%%%%%%%%%%%

\begin{minipage}[t]{0.33\textwidth}

%%%%%%%%%%%%%%%%%%%%%%%%%%%%%%%%%%%%%%
%     EDUCATION
%%%%%%%%%%%%%%%%%%%%%%%%%%%%%%%%%%%%%%

\section{Education}

\subsection{University of Texas}

\vspace{\topsep} % Hacky fix for awkward extra vertical space

\descript{M.S. in Computer Science}
\location{May 2017 | Austin, TX}
College of Natural Science \\
Five Years BS/MS Integrated Program \\
\location{Major GPA: 3.81 / 4.0}
\sectionsep

\descript{B.S. in Computer Science}
\location{May 2016 | Austin, TX}
College of Natural Science \\
% Turing Scholars Program \\
\location{Major GPA: 3.95 / 4.0}
\sectionsep


%%%%%%%%%%%%%%%%%%%%%%%%%%%%%%%%%%%%%%
%     COURSEWORK
%%%%%%%%%%%%%%%%%%%%%%%%%%%%%%%%%%%%%%

\section{Courses}

\subsection{Undergraduate}
Operating System \\
Algorithm \& Complexity \\
Artificial Intelligence \\
Programming Languages \\
Computer Vision/Machine Learning \\
Data Mining \\
Network \& Privacy \\

\sectionsep

\subsection{Graduate}
Compiler \\
Computer Graphics \\
Autonomous Robots \\
Software Design \\
Advanced Operating System \\
Numerical Linear Algebra \\

\sectionsep

%%%%%%%%%%%%%%%%%%%%%%%%%%%%%%%%%%%%%%
%     SKILLS
%%%%%%%%%%%%%%%%%%%%%%%%%%%%%%%%%%%%%%

\section{Skills}
\subsection{Languages}
\begin{tabular}{ll}
\skillbar{C/C++\hspace{1.5em}}{0.55}
\skillbar{Java}{0.75}
\skillbar{Python}{0.75}
\end{tabular}

\sectionsep

\subsection{Interests}
\begin{tabular}{ll}
\skillbar{Graphics}{0.75}
\skillbar{System}{0.5}
\skillbar{Compiler}{0.4}
\skillbar{Web}{0.6}
\end{tabular}

\sectionsep

%%%%%%%%%%%%%%%%%%%%%%%%%%%%%%%%%%%%%%
%     LINKS
%%%%%%%%%%%%%%%%%%%%%%%%%%%%%%%%%%%%%%

\section{Links}
Github: \href{https://github.com/tycheng}{\custombold{tycheng}} \\
LinkedIn: \href{www.linkedin.com/pub/tianyu-cheng/66/335/834/}{\custombold{tianyu-cheng}} \\
Homepage: \href{http://tycheng.github.io}{\custombold{tycheng.github.io}} \\

\sectionsep

%%%%%%%%%%%%%%%%%%%%%%%%%%%%%%%%%%%%%%
%
%     COLUMN TWO
%
%%%%%%%%%%%%%%%%%%%%%%%%%%%%%%%%%%%%%%

\end{minipage}
\hfill
\begin{minipage}[t]{0.66\textwidth}

%%%%%%%%%%%%%%%%%%%%%%%%%%%%%%%%%%%%%%
%     EXPERIENCE
%%%%%%%%%%%%%%%%%%%%%%%%%%%%%%%%%%%%%%

\section{Experience}

\runsubsection{Apple}
\descript{| GPU Architectural Validation Team }
\location{Jun 2017 – Current | Austin, TX}
\vspace{\topsep} % Hacky fix for awkward extra vertical space
\begin{tightemize}
\item create software to verify architectural and micro-architectural functionality,
    performance, and power of pre-silicon hardware designs
\item review specifications, develop attributes, tests and coverage plans, and define
    methodology and test benches
\end{tightemize}
\sectionsep

\runsubsection{University of Texas at Austin}
\descript{| Research Assistant }
\location{Jan 2017 – May 2017 | Austin, TX}
% \vspace{\topsep} % Hacky fix for awkward extra vertical space
\begin{tightemize}
\item measured performance of non-volatile memory libraries
\item port git to use transactional file system
\end{tightemize}
\sectionsep

\runsubsection{Apple}
\descript{| GPU Architectural Validation Team }
\location{May 2016 – August 2016 | Austin, TX}
% \vspace{\topsep} % Hacky fix for awkward extra vertical space
\begin{tightemize}
\item developed an internal web front-end tool for performance visualization
\item implemented and validated counters in performance model
\item worked on numerics validation for GPU driver
\end{tightemize}
\sectionsep

\runsubsection{Apple}
\descript{| GPU Architectural Validation Team }
\location{May 2015 – August 2015 | Austin, TX}
% \vspace{\topsep} % Hacky fix for awkward extra vertical space
\begin{tightemize}
\item developed an internal server-side tool with Ruby on Rails for test automation
\item developed a web front-end data analysis tool for data visualization
\item worked on numerics validation for GPU driver
\end{tightemize}
\sectionsep

% \runsubsection{Digital Media Institute}
% \descript{| Student Technician }
% \location{June 2014 – December 2014 | Austin, TX}
% % \vspace{\topsep} % Hacky fix for awkward extra vertical space
% \begin{tightemize}
% \item back-end design and implementation for an educational game in Unity
% \item developed several third-party tools to facilitate game development
% \end{tightemize}
% \sectionsep

%%%%%%%%%%%%%%%%%%%%%%%%%%%%%%%%%%%%%%
%     Publications
%%%%%%%%%%%%%%%%%%%%%%%%%%%%%%%%%%%%%%

\section{Publications}

\runsubsection{TxFS}
\descript{ Leveraging File-System Crash Consistency to Provide ACID Transactions }
\location{USENIX ATC '18}
\begin{tightemize}
\item Yige Hu, Zhiting Zhu, Ian Neal, Youngjin Kwon, and Tianyu Cheng, The University of Texas at Austin
\item Vijay Chidambaram, The University of Texas at Austin and VMware Research; Emmett Witchel, The University of Texas at Austin
\end{tightemize}
\sectionsep

%%%%%%%%%%%%%%%%%%%%%%%%%%%%%%%%%%%%%%
%     Projects
%%%%%%%%%%%%%%%%%%%%%%%%%%%%%%%%%%%%%%

\section{Projects}

\runsubsection{Planet Render}
\descript{| Computer Graphics }
\begin{tightemize}
\item a procedural terrain rendering program in OpenGL/GLSL
\item procedural terrain generation based on Perlin noise
\item LoD (level-of-detail) terrain/ocean rendering with CDLOD (continuous-distance LoD)
\end{tightemize}
\sectionsep

\runsubsection{Ray Tracer}
\descript{| Computer Graphics }
\begin{tightemize}
\item a multithreaded ray tracer based on Whitted model
\item used KD-tree and SAH for ray-object intersection optimization
\item supports glossiness and depth of field using distribution ray tracing
\end{tightemize}
\sectionsep

% \runsubsection{3D Animator}
% \descript{| Computer Graphics Animation}
% \begin{tightemize}
% \item a simple 3D animator for 3D format PMD/PMX
% \item implemented using OpenGL/GLSL for model rendering
% \item supports FK (forward kinematics) and simple IK (inverse kinematics)
% \end{tightemize}
\sectionsep

\runsubsection{Latte Compiler}
\descript{| Deep Learning Compiler }
\begin{tightemize}
\item a source-to-source compiler for deep neural network %in Python-style descriptive language
\item AST pattern match for parsing deep neural network architecture
\item loop structure optimization with Intel MKL(BLAS) library
\item data structure transformation for cache optimization
\end{tightemize}
\sectionsep

% \runsubsection{GameL}
% \descript{| Scala DSL }
% \begin{tightemize}
% \item a game scripting DSL(domain-specific language) using Scala and Swing
% \item designed syntax for basic game object creation and manipulation
% \item implemented a demo of the classical game Snake using GameL
% \end{tightemize}
% \sectionsep

\end{minipage}
\end{document}  \documentclass[]{article}
